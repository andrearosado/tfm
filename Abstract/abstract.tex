% ************************** Abstract *****************************
% Use `abstract' as an option in the document class to print only the titlepage and the abstract.

\begin{abstract}

The existence of increasingly large and voluminous spatial databases leads to different challenges in the sense of managing and providing an agile and direct access to their information. The database of the Information System on Land Occupation of Spain (SIOSE) is a clear example of this situation, when presenting information of complex coding and tens of millions of records, demanding from its users a great capacity for processing and storage. Among the large number of possible applications from databases such as the SIOSE, landscape metrics are useful for analyzing the structure and behavior of the landscape. Currently, there are many softwares designed to offer calculations and analysis of landscape patterns from databases of land occupation (FRAGSTATS, Patch Analyst, etc). However, no program can calculate all the metrics nor is it designed to work on current, more complex and voluminous geodatabases, which refers to the \textit{usability} problems related to the SIOSE. The main objective of this work is to create a PostgreSQL / PostGIS extension capable of calculating landscape metrics from the SIOSE database, dealing with \textit{usability} problems. The central question of this work is whether this extension will be able to cope with the aforementioned \textit{usability} problems that affect the SIOSE and other similar geodatabases.\\


\textbf{Key words}: SIOSE, \textit{usability}, landscape metrics, PostGIS, extension.


\end{abstract}
% ************************** Abstract *****************************
% Use `abstract' as an option in the document class to print only the titlepage and the abstract.

\begin{abstract}

The spatial databases are increasingly complex and voluminous, generating new challenges in the use of geographic information. The data base of the Sistema de Información sobre Ocupación del Suelo de España (SIOSE) is a clear example of this trend, since it implements a relatively complex data model and collects tens of millions of records with territorial information.

Among the possible applications of databases such as the SIOSE, landscape metrics serve to analyze the structure of the landscape and other related phenomena. However, given the great diversity of possible metrics, no \textit{software} allows you to calculate them all, nor is it designed to work with geodatabases as complex and voluminous as the SIOSE.
 
The main objective of this work is to create a database extension on PostgreSQL/PostGIS that calculates landscape metrics from the SIOSE database, addressing the aforementioned \textit{usability} problems. In this sense, the central question of this work is whether this extension will be able to face the problems of \textit{usability} that affect the SIOSE and other similar geodatabases.

This work has contributed to the development of the extension \pgland{}, which calculates landscape metrics from simple SQL statements. The extension has been put to the test in a computational experience based on SIOSE-2011 data. It has been proved that \pgland{} allows us to answer hundreds of queries about landscape metrics in a few seconds, which suggests that this type of calculations could be offered as a web query service, thus solving the problems of \textit{usability}.\\

\textbf{Keywords}: SIOSE, \textit{usability}, landscape metrics, PostGIS, reproducibility, containerization.



\end{abstract}
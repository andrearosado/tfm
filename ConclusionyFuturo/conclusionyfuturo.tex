%!TEX root = ../thesis.tex
%*******************************************************************************
%****************************** Third Chapter **********************************
%*******************************************************************************
\chapter{Conclusiones y trabajo futuro}\label{chap:concl}

\begin{graybox}
\begin{itemize}
\item Se ha desarrollado una extensión que simplifica complejas consultas SQL a consultas de una o pocas líneas.
\item El rendimiento de las consultas resulta prometedor, lo que permitirá crear servicios web de consulta directa sobre el SIOSE y bases de datos similares.   
\item El trabajo con \textit{dockers} facilita la reproducibilidad de la investigación y un despliegue escalable en Internet.
\item Los conocimientos adquiridos en el Máster han servido como introducción a un campo profesional muy complejo.
\end{itemize}
\end{graybox}

Los objetivos planteados en la \nameref{chap:intro} comprendian aspectos de trabajo colaborativo, cuestiones tecnológicas y había una gran preocupación por mejorar la usabilidad de bases de datos voluminosas y complejas como la del SIOSE. En este sentido, el objetivo principal se ha conseguido al contribuir significativamente en el desarrollo de una extensión de Postgres/PostGIS realmente potente, utilizando las técnologías y plataformas más actuales.

El desarrollo de nuevas métricas se puede sistematizar en gran medida, lo que permite repartir el trabajo en equipos multidisciplinares, como lo es el del Laboratorio de Geomática de la Universidad de Alicante.

Este laboratorio trabaja con las mismas herramientas que utilizan algunas empresas de referencia (CARTO, Geographica, etc) 

En este trabajo se ha conseguido desarrollar un prototipo de una aplicación que, según los objetivos del Proyecto SIOSE-INNOVA, se debería desarrollar en unos tres años. El desarrollo de una extensión \textit{en producción} llevará más tiempo, es un trabajo complejo que requiere de todo un equipo de expertos y meses de trabajo. \textbf{El trabajo en equipo es esencial en este tipo de proyectos.}

Tras redactar este trabajo es posible valorar aún más los contenidos del ''Máster en Tecnologías de la Información Geográfica para la Ordenación del Terrritorio: SIG y Teledetección''. En las asignaturas de \textit{fundamentos teóricos sobre bases de datos, diseño e implementación de bases de datos, lenguajes de programación, software libre, análisis espacial básico, análisis visual de imágenes y desarrollo e implementación de la información geográfica en aplicaciones infográficas} se adquirieron conocimientos básicos para empezar a trabajar en un proyecto sobre geodatabases como este. Esta experiencia será similar en otro tipo de proyectos sobre teledetección, cartografía, etc, con lo cual el aprendizaje será constante en cualquier rama de las TIG.

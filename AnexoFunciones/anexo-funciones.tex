%!TEX root = ../thesis.tex
% ******************************* Thesis Appendix A ****************************
\chapter{Funciones tipo}\label{anex:funcionestipo}

<-------Mover texto de los anexos a resultados-------->

En el anexo \ref{anex:funcionestipo} se presentan los ejemplos más básicos de funciones para calcular los distintos tipos de métricas. Se pueden distinguir métricas que requieren un único registro (\textbf{\textit{funciones SQL}}), frente a aquellas funciones que necesitan considerar más de un registro (\textbf{\textit{funciones agregadas}}).

La programación puede variar en complejidad. Por ejemplo, hay funciones agregadas que aplican cálculos posteriores a la agregación de valores o funciones que realizan más de un cálculo simultáneo. Además, las métricas no siempre se calculan igual si son proyectadas (\textit{geometry}) o en coordenadas esféricas (\textit{geography}).

Los ejemplos más sencillos de los distintos tipos de métricas serían las que se calculan únicamente a partir de las áreas de uno o más polígonos: \textit{Patch Area}, \textit{Class Area} y \textit{Landscape Area}. La implementación del resto de métricas puede consultarse en el repositorio del proyecto en GitHub.

\lstset{caption=Función para calcular AREA de un polígono de PostGIS de tipo \textit{geography},label= p_area}
\begin{SQL}
CREATE OR REPLACE FUNCTION p_area(geom geography)
RETURNS metric AS 
$$

SELECT (1, St_Area(geom)/10000)::metric;

$$
LANGUAGE SQL
IMMUTABLE
RETURNS NULL ON NULL INPUT;

COMMENT ON FUNCTION p_area(geom geography) IS 'Divide el área en metros cuadrados de un polígono por 10.000 para devolver un valor en Hectáreas.';
\end{SQL}
\pagebreak


\lstset{caption=Crear una función para calcular el IDW (I),label= IDW1}
\begin{SQL}
SELECT SUM(St_Area(geom))/10000, category FROM sample_patches_25830 GROUP BY category;
SELECT SUM(St_Area(geom))/10000, category FROM sample_patches_4326 GROUP BY category;
\end{SQL}
\pagebreak

\lstset{caption=Crear una función para calcular el IDW (I),label= IDW1}
\begin{SQL}
SELECT SUM(St_Area(geom)) FROM sample_patches_25830;
SELECT SUM(St_Area(geom)) FROM sample_patches_4326;
\end{SQL}
\pagebreak

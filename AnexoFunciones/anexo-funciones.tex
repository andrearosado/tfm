%!TEX root = ../thesis.tex
% ******************************* Thesis Appendix A ****************************
\chapter{Funciones tipo}\label{anex:funcionestipo}

<-------Mover texto de los anexos a resultados-------->

En el anexo \ref{anex:funcionestipo} se presentan los ejemplos más básicos de funciones para calcular los distintos tipos de métricas. Se pueden distinguir métricas que requieren un único registro (\textbf{\textit{funciones SQL}}), frente a aquellas funciones que necesitan considerar más de un registro (\textbf{\textit{funciones agregadas}}).

La programación puede variar en complejidad. Por ejemplo, hay funciones agregadas que aplican cálculos posteriores a la agregación de valores o funciones que realizan más de un cálculo simultáneo. Además, las métricas no siempre se calculan igual si son proyectadas (\textit{geometry}) o en coordenadas esféricas (\textit{geography}).

Los ejemplos más sencillos de los distintos tipos de métricas serían las que se calculan únicamente a partir de las áreas de uno o más polígonos: \textit{Patch Area}, \textit{Class Area} y \textit{Landscape Area}. La implementación del resto de métricas puede consultarse en el repositorio del proyecto en GitHub.

\lstset{caption=Función para calcular AREA de un polígono de PostGIS de tipo \textit{geography},label= p_area}
\begin{SQL}
CREATE OR REPLACE FUNCTION p_area(geom geography)
RETURNS metric AS 
$$

SELECT (1, St_Area(geom)/10000)::metric;

$$
LANGUAGE SQL
IMMUTABLE
RETURNS NULL ON NULL INPUT;

COMMENT ON FUNCTION p_area(geom geography) IS 'Divide el área en metros cuadrados de un polígono por 10.000 para devolver un valor en Hectáreas.';
\end{SQL}
\pagebreak


\lstset{caption=Función para calcular CA por categorías.,label= c_area}
\begin{SQL}
/*
Total (Class) Area - devuelve la suma de las áreas (m²) de todos los polígonos correspondientes al tipo de polígono, dividido por 10.000 (unidades: Hectáreas).
*/
-- SAMPLE USAGE
/*
SELECT c_totalarea_state(ARRAY[('Agricultural area',('Total Class Area'::text, 10, 'Ha.'::text))::labeled_metric], geom, category) FROM sample_patches_25830;
SELECT c_totalarea_state(ARRAY[('Agricultural area',('Total Class Area'::text, 10, 'Ha.'::text))::labeled_metric], geom, category) FROM sample_patches_4326;
*/

CREATE OR REPLACE FUNCTION c_totalarea_state(
	current_state metric_labeled,
	geom geometry,
	category text)
    RETURNS metric_labeled
    LANGUAGE 'sql'

AS 
$BODY$

WITH inputs AS (
    	SELECT current_state AS cstate
), melt AS (
    	SELECT unnest((cstate).pairs) AS m2 FROM inputs
    	UNION 
	SELECT (category, (p_area(geom)).value)::metric_labeled_pair AS m2
), summarize AS (
	SELECT (m2).label, SUM((m2).value) AS value FROM melt GROUP BY (m2).label
)
SELECT (9, ARRAY(SELECT (label, value)::metric_labeled_pair FROM summarize))::metric_labeled;

$BODY$;


-- SAMPLE USAGE
-- SELECT c_totalarea(geom,category) FROM sample_patches_25830;

CREATE AGGREGATE c_totalarea(geometry, text)(
    SFUNC=c_totalarea_state,
    STYPE=metric_labeled,
    INITCOND='(9,{})'
);

COMMENT ON AGGREGATE c_totalarea(geom geometry, category text) IS 'Calcula la suma de las áreas de los polígonos de la misma categoría dividido por 10.000 para devolver un valor en Hectáreas.';



CREATE OR REPLACE FUNCTION c_totalarea_state(
	current_state metric_labeled,
	geom geography,
	category text)
    RETURNS metric_labeled
    LANGUAGE 'sql'

AS 
$BODY$

WITH inputs AS (
    	SELECT current_state AS cstate
), melt AS (
    	SELECT unnest((cstate).pairs) AS m2 FROM inputs
    	UNION 
	SELECT (category, (p_area(geom)).value)::metric_labeled_pair AS m2
), summarize AS (
	SELECT (m2).label, SUM((m2).value) AS value FROM melt GROUP BY (m2).label
)
SELECT (9, ARRAY(SELECT (label, value)::metric_labeled_pair FROM summarize))::metric_labeled;

$BODY$;


-- SAMPLE USAGE
-- SELECT c_totalarea(geom,category) FROM sample_patches_4326;

CREATE AGGREGATE c_totalarea(geography, text)(
    SFUNC=c_totalarea_state,
    STYPE=metric_labeled,
    INITCOND='(9,{})'
);

COMMENT ON AGGREGATE c_totalarea(geom geography, category text) IS 'Calcula la suma de las áreas de los polígonos de la misma categoría dividido por 10.000 para devolver un valor en Hectáreas.';
\end{SQL}
\pagebreak

\lstset{caption=Función para calcular TA del paisaje.,label= IDW1}
\begin{SQL}
/*
Total Area - devuelve el total del área (m²) del paisaje dividido por 10.000 (unidades: Hectáreas).
*/
-- SAMPLE USAGE: 
/*
SELECT (l_totalarea(geom)).value FROM sample_patches_25830;
SELECT (l_totalarea(geom)).value FROM sample_patches_4326;
*/

CREATE OR REPLACE FUNCTION l_totalarea_state(metric,geometry)
    RETURNS metric AS
$$
	SELECT $1 + (p_area($2)).value; 
$$
LANGUAGE 'sql' IMMUTABLE;


CREATE AGGREGATE l_totalarea(geometry)(
    SFUNC=l_totalarea_state,
    STYPE=metric,
    INITCOND='(0,0)'
);

COMMENT ON AGGREGATE l_totalarea(geometry) IS 'Calcula el área total del paisaje dividida por 10.000 para devolver un valor en Hectáreas.';


CREATE OR REPLACE FUNCTION l_totalarea_state(metric,geography)
    RETURNS metric AS
$$
	SELECT $1 + (p_area($2)).value;
$$
LANGUAGE 'sql' IMMUTABLE;


CREATE AGGREGATE l_totalarea(geography)(
    SFUNC=l_totalarea_state,
    STYPE=metric,
    INITCOND='(0,0)'
);

COMMENT ON AGGREGATE l_totalarea(geography) IS 'Calcula el área total del paisaje dividida por 10.000 para devolver un valor en Hectáreas.';
\end{SQL}
\pagebreak

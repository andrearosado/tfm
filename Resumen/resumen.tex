% ************************** Resumen *****************************
% Use `abstract' as an option in the document class to print only the titlepage and the abstract.

\begin{resumen}

La existencia de bases de datos espaciales cada vez más amplias y voluminosas conlleva a distintos retos en el sentido de manejar y proporcionar un acceso ágil y directo de sus informaciones. La base de datos del Sistema de Información sobre Ocupación del Suelo de España (SIOSE) es un claro ejemplo de esta situación, al presentar información de compleja codificación y decenas de millones de registros, exigiendo de sus usuarios una gran capacidad de procesamiento y almacenamiento. Entre el gran número de aplicaciones posibles a partir de bases de datos como la del SIOSE, las métricas del paisaje son útiles para analizar la estructura y comportamiento del paisaje. Actualmente, existen muchos softwares diseñados para ofrecer cálculos y análisis de patrones de paisaje a partir de bases de datos de ocupación del suelo (FRAGSTATS, Patch Analyst, etc). Sin embargo, ningún programa permite calcular todas las métricas ni está pensado para trabajar sobre geodatabases actuales, más complejas y voluminosas, lo cual se refiere a los problemas de \textit{usabilidad} relacionados con el SIOSE. El objetivo principal de este trabajo es crear una extensión PostgreSQL/PostGIS capaz de calcular métricas del paisaje a partir de la base de datos del SIOSE, haciendo frente a los problemas de \textit{usabilidad}. La pregunta central de este trabajo es si esta extensión será capaz de hacer frente a los mencionados problemas de \textit{usabilidad} que afectan al SIOSE y a otras geodatabases similares.\\


\textbf{Palabras clave}: SIOSE, \textit{usabilidad}, métricas de paisaje, PostGIS, extensión.


\end{resumen}

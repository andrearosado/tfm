% ************************** Resumen *****************************
% Use `abstract' as an option in the document class to print only the titlepage and the abstract.

\begin{resumen}

La existencia de base de datos espaciales cada vez más amplias y voluminosas conlleva a distintos retos en el sentido de manejar y proporcionar un acceso ágil y directo de sus informaciones. La base de dados del Sistema de Información sobre Ocupación del Suelo de España (SIOSE) es un claro ejemplo de esta situación, al presentar informaciones de compleja codificación y decenas de millones de registros, exigiendo de sus usuarios una gran capacidad de procesamiento y almacenamiento.

Este trabajo se enmarca dentro de los objetivos propuestos por el Proyecto SIOSE-INNOVA que plantea dos líneas de trabajo, una de innovación técnica y otra aplicada, que buscan potenciar la usabilidad del SIOSE en distintos tipos de estudio.

Entre el gran número de aplicaciones posibles a partir de bases de datos como la del SIOSE, las métricas del paisaje son útiles para analizar la estructura y comportamiento del paisaje. Sin embargo, dada la gran diversidad de métricas, ningún programa permite calcular todas las métricas ni está pensado para trabajar con geodatabases tan complejas y voluminosas como es la del SIOSE la cual se refiere también a problemas de usabilidad.
 
El objetivo principal de este trabajo es crear una extensión sobre PostgreSQL/PostGIS capaz de calcular métricas de paisaje a partir de la base de datos del SIOSE, haciendo frente a los problemas de usabilidad. La pregunta central de este trabajo es si esta extensión será capaz de hacer frente a los mencionados problemas que afectan al SIOSE y geodatabases similares. 

El desarrollo de esta extensión, denominada \pgland{}, ha sido posible gracias al uso de herramientas de desarrollo colaborativo y plataformas de contenerización o virtualización de servicios. Han permitido establecer flujos de trabajo de integración continua e ir generando una extensión funcional capaz de calcular métricas de paisaje a partir del SIOSE-2011. Se han utilizado toda una variedad de técnicas de programación en bases de datos para que los usuarios sean capaces de calcular métricas de paisaje en sentencias de una o unas pocas líneas. Finalmente, para poner a prueba la extensión, se ha llevado a cabo una experiencia computacional sobre el SIOSE-2011 a partir de repetidas consultas de métricas. 

Este trabajo sirve para poner en valor el trabajo colaborativo basado en una serie de herramientas de control de versiones, contenerización y orquestación. Esta metodología facilitará enormemente a seguir añadiendo nuevas métricas y aplicar esta extensión a nuevos estudios relacionados con la estructura del paisaje.\\


\textbf{Palabras clave}: SIOSE, \textit{usabilidad}, métricas de paisaje, PostGIS, reproducibilidad, contenerización.


\end{resumen}

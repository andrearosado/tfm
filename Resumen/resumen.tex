% ************************** Resumen *****************************
% Use `abstract' as an option in the document class to print only the titlepage and the abstract.

\begin{resumen}

\textbf{Este trabajo demuestra que es posible hacer un uso más ágil y directo de las últimas bases de datos de ocupación del suelo}. Estas bases de datos son cada vez más voluminosas, por lo que se ven en la necesidad de utilizar modelos de datos relativamente complejos para manejar tanta información. En este caso se ha analizado esta cuestión sobre el caso del SIOSE, pero las tendencias en este campo hacen que cuanto se desarrolla en las proximas páginas sea extrapolable a otras geodatabases europeas e internacionales.

SIOSE es el Sistema de Información sobre Ocupación del Suelo de España. La base de datos del SIOSE contiene un gran volumen de información (decenas de millones de registros) con un gran potencial de aplicación. Además, esta base de datos implementa un modelo de datos \textit{orientado a objetos} que aporta una \textbf{mayor riqueza descriptiva de la que se da en otras bases de datos de ocupación del suelo anteriores}. Así pues, mientras que en otras bases de datos se da una correspondencia de una etiqueta descriptiva por cada polígono, en el SIOSE cada polígono puede contener toda la información descriptiva que sea necesaria (decenas de etiquetas descriptivas o más). No obstante, \textbf{esta gran cantidad de información y la complejidad del modelo de datos del SIOSE, se manifiestan en problemas de \textit{usabilidad} de esta información por parte de los usuarios}. Cabe anticipar que, según aumenta la capacidad para generar nueva información geográfica, dichos problemas irán en aumento en los próximos años.

Este trabajo se enmarca dentro de los objetivos propuestos por el Proyecto SIOSE-INNOVA (CSO2016-79420-R). Este proyecto plantea dos líneas de trabajo complementarias, una de innovación técnica y otra aplicada, que buscan potenciar la \textit{usabilidad} del SIOSE en distintos tipos de estudios (biodiversidad, riesgos naturales, análisis paisajístico, estudios de accesibilidad, etc).

Entre el gran número de aplicaciones posibles a partir de bases de datos como la del SIOSE, \textbf{las métricas del paisaje son utiles para analizar la estructura y comportamiento del paisaje}. Actualmente, existen muchos \textit{softwares} diseñados para ofrecer cálculos y análisis de patrones de paisaje a partir de bases de datos de ocupación del suelo (FRAGSTATS, Patch Analyst, etc). Sin embargo, dada la gran diversidad de métricas posibles, ningún programa permite calcular \textit{todas} las métricas, ni está pensado para trabajar con las geodatabases actuales, más complejas y voluminosas que hace unos años, lo cual se refiere también a los problemas de \textit{usabilidad} relacionados con el SIOSE (volumen y complejidad).

\textbf{El objetivo principal de este trabajo es crear una extensión sobre PostgreSQL/PostGIS capaz de calcular métricas del paisaje a partir de la base de datos del SIOSE}, haciendo frente a los mencionados problemas de \textit{usabilidad} (volumen de datos y complejidad de consultas). \textbf{La pregunta central de este trabajo es si esta extensión} (en combinación con otras nuevas tecnologías), \textbf{será capaz de hacer frente a los mencionados problemas de \textit{usabilidad} que afectan al SIOSE y a otras geodatabases similares.}

El desarrollo de esta extensión, denominada \textit{pg\_landmetrics}, ha sido posible gracias al uso de herramientas de desarrollo colaborativo (\textit{Git, GitHub}) y al uso de plataformas de contenerización o virtualización de servicios (\textit{Docker, DockerHub}). Estas herramientas han permitido establecer flujos de trabajo de \textit{integración continua} e ir generando, paso a paso, una extensión funcional que es capaz de calcular métricas del paisaje a partir de la base de datos del SIOSE-2011. Se han utilizado toda una variedad de técnicas de programación en bases de datos para que los usuarios de esta extensión sean capaces de calcular un gran número de métricas del paisaje en sentencias de una o unas pocas líneas.

Finalmente, para poner a prueba la extensión desarrollada se ha llevado a cabo \textbf{una experiencia computacional completa sobre la base de datos del SIOSE-2011}. Esta experiencia ha consistido en simular repetidas consultas de métricas por parte de un usuario sobre dos grandes áreas y a distintas escalas (ver Figura \ref{fig:visorweb}). Dado que estas consultas se realizan en pocos segundos en un PC, \textbf{los resultados son prometedores} y es seguro que mejorarán cuando este \textit{software} funcione en un servidor de Internet o sea distribuido en \textit{La Nube}.

Globalmente, este trabajo sirve para poner en valor el \textbf{trabajo colaborativo} basado en una serie de herramientas de control de versiones, contenerización y orquestación, que pueden ser aplicadas en distintos contextos. Esta metodología facilitará enormemente seguir añadiendo nuevas métricas del paisaje y aplicar esta extensión en nuevos estudios relacionados con la estructura del paisaje.\\


\textbf{Palabras clave}: SIOSE, \textit{usabilidad}, métricas de paisaje, PostGIS, reproducibilidad, contenerización


\end{resumen}

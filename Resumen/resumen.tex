% ************************** Resumen *****************************
% Use `abstract' as an option in the document class to print only the titlepage and the abstract.

\begin{resumen}

Las bases de datos espaciales son cada vez más complejas y voluminosas, generando nuevos retos en el aprovechamiento de la información geográfica. La base de dados del Sistema de Información sobre Ocupación del Suelo de España (SIOSE) es un claro ejemplo de esta tendencia, ya que implementa un modelo de datos relativamente complejo y recoge decenas de millones de registros con información territorial.

Entre las aplicaciones posibles de las bases de datos como la del SIOSE, las métricas del paisaje sirven para analizar la estructura del paisaje y otros fenómenos relacionados. Sin embargo, dada la gran diversidad de métricas posibles, ningún \textit{software} permite calcularlas todas, ni está pensado para trabajar con geodatabases tan complejas y voluminosas como la del SIOSE.
 
El objetivo principal de este trabajo consiste en crear una extensión de bases de datos sobre PostgreSQL/PostGIS que calcule métricas del paisaje a partir de la base de datos del SIOSE, haciendo frente a los mencionados problemas de \textit{usabilidad}. En este sentido, la pregunta central de este trabajo es si esta extensión será capaz de enfrentar los problemas de \textit{usabilidad} que afectan al SIOSE y a otras geodatabases similares.

En este trabajo se ha contribuido en el desarrollo de la extensión \pgland{}, que calcula métricas de paisaje a partir de sencillas sentencias SQL. La extensión ha sido puesta a prueba en una experiencia computacional sobre la base de datos del SIOSE-2011. Se ha comprobado que \pgland{} permite responder a centenares de consultas sobre métricas del paisaje en pocos segundos, con lo que se plantea que este tipo de cálculos podrían ser ofrecidos como un servicio web de consultas, resolviendo así los problemas de \textit{usabilidad}.\\


\textbf{Palabras clave}: SIOSE, \textit{usabilidad}, métricas de paisaje, PostGIS, reproducibilidad, contenerización.


\end{resumen}

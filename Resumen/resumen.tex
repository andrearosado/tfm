% ************************** Resumen *****************************
% Use `abstract' as an option in the document class to print only the titlepage and the abstract.

\begin{resumen}
Las métricas de paisaje se utilizan para analizar la estructura y comportamiento del paisaje como también las modificaciones temporales, ya sea por factores naturales o humanos. Dada la utilidad para una variedad de aplicaciones, existen muchos softwares diseñados para ofrecer cálculos y análisis de patrones de paisaje. El objetivo principal de este trabajo es crear una extensión PostgreSQL/PostGIS reproducible y extensible capaz de calcular métricas de paisaje para datos de entrada vectoriales. Más adelante, esta extensión debería permitir añadir nuevas métricas e investigar nuevos estudios relacionados con la estructura del paisaje. Finalmente, se pone en valor la implementación de la extensión ya que requiere de una metodología de trabajo colaborativo basado en una serie de herramientas de contenerización y orquestación.\\


\textbf{Palabras clave}: métricas de paisaje, extensión, reproducibilidad, contenerización, orquestación.


\end{resumen}

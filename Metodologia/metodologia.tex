%!TEX root = ../thesis.tex
%*******************************************************************************
%****************************** Metodologia *********************************
%*******************************************************************************

\chapter{Metodología}
El desarrollo del software requiere de una metodología de trabajo colaborativo basado en una serie de herramientas de contenerización y orquestación de manera eficiente, extensible y reproducible.

\subsection*{2.1. Software y plataformas}

\subsection*{2.2. Conjunto de datos}

Antes de seleccionar las métricas de paisaje que han sido utilizadas para complementar el nuevo software, ha sido necesario elaborar y procesar conjuntos de datos que posteriormente se han utilizado para realizar comprobaciones del funcionamiento de las métricas. Así pues, por una parte se ha creado un paisaje ficticio y por otra, se han utilizado los datos del SIOSE de dos zonas de estudio como casos de estudio reales.

Para el paisaje ficticio, se ha digitalizado desde cero todos los polígonos que comprenden esta zona de estudio a partir de herramientas de geoprocesamiento, geometría, vectorial y edición, utilizando QGIS. En la tabla de atributos del shapefile, cada polígono tiene identificador único, geometría, clasificación según el tipo de cobertura del suelo y color según el tipo de categoría al que pertenece.

% Please add the following required packages to your document preamble:
% \usepackage{booktabs}
\begin{table}[]
\centering
\caption{. Atributos del primer conjunto de datos.}
\label{my-label}
\begin{tabular}{@{}lll@{}}
\toprule
\textbf{Nombre} & \textbf{Tipo de campo}   & \textbf{Descripción}                    \\ \midrule
gid             & integer                  & Identificador único de cada polígono    \\
geom            & geometry                 & Geometría del polígono                  \\
category        & character varying (text) & Clasificación de la cobertura del suelo \\
svg\_color      & character varying (text) & Color según tipo de categoría           \\ \bottomrule
\end{tabular}
\end{table}


\subsection*{2.3. Selección de métricas}

\subsection*{2.4. Implementación/desarrollo de funciones en PostgreSQL}

\subsection*{2.5. Documentación de la extensión}

A lo largo del trabajo, se aplican una serie de lenguajes de marcado para la documentación de la extensión. Pero antes se obtienen conocimientos previos sobre ellos y su funcionamiento para utilizarlos durante el proyecto. Los lenguajes utilizados son:
\begin{itemize}
\item\textbf{Markdown}\footnote{\url{https://github.com/adam-p/markdown-here/wiki/Markdown-Cheatsheet}} es un lenguaje ligero capaz de convertir texto plano a lenguaje HTML. Permite una escritura sencilla y conserva un diseño fácil de lectura. Es compatible con muchas plataformas. Este tipo de lenguaje es utilizado para documentar la extensión en la plataforma de GitHub y la descripción de cada una de las métricas de paisaje.

\item\textbf{TeX}\footnote{\url{https://www.latex-project.org/}} es el lenguaje que se utiliza en el sistema de textos LaTeX y que crea documentos con una alta calidad tipográfica. Se utiliza para escribir artículos o libros científicos, y desde hace tiempo este lenguaje se emplea por un gran número de usuarios. Este tipo de lenguaje es utilizado para redactar este trabajo. Para trabajar con este lenguaje, se utiliza la aplicación Texmaker que se ejecuta desde un contenedor Docker.

\item\textbf{Scalable Vector Graphics (SVG)} es un lenguaje capaz de crear gráficos basados en vectores escalables a partir de archivos vectoriales en 2D y en formato XML. En esta década muchos de los navegadores web utilizan este tipo de lenguaje para sus gráficos. Gracias a este lenguaje, los gráficos no pierden calidad, pueden ser escalables y ocupan menos espacio en la memoria. Este tipo de lenguaje es utilizado para crear las figuras del trabajo.

Por ello, se desarolla una función SQL, como primera propuesta, que dibuja y colorea los gráficos vectoriales que acompañan en el trabajo y en la documentación de la extensión en la plataforma GitHub. Para que las figuras tengan el color correspondiente al tipo de cobertura al que pertenecen, se aplica el código SVG que acompaña a cada polígono en la tabla de atributos del primer conjuntos de datos. La función SVG puede consultarse en el Anexo II.

\item\textbf{Mermaid}\footnote{\url{https://mermaidjs.github.io/}} es un lenguaje de secuencia, que utiliza etiquetas similares a las que son empleadas en el lenguaje de marcado, capaz de generar gráficos a partir de texto por medio de JavaScript. Este tipo de lenguaje se ha utilizado para crear los diagramas de secuencia y el diagrama de Gantt.
\end{itemize}



%!TEX root = ../thesis.tex
%*******************************************************************************
%****************************** Metodologia *********************************
%*******************************************************************************


\chapter{Metodología}\label{chap:metod}

Este trabajo aplica una metodología de \textbf{integración continua y desarrollo colaborativo} de acuerdo con los objetivos del proyecto SIOSE-INNOVA.

La implementación de \textit{pg\_landmetrics} requiere de una metodología de trabajo colaborativo basado en una serie de herramientas de contenerización y orquestación de manera eficiente, extensible y reproducible.


\section{Integración continua y desarrollo colaborativo}

\begin{graybox}
\begin{itemize}
\item El trabajo colaborativo se ha coordinado utilizando \textit{Git} que es el sistema de \textbf{control de versiones} más popular de los últimos años (p.ej. utilizado en , PostGIS, QGIS, CARTO y decenas de proyectos ESRI, entre muchos otros).
\item La \textbf{contenerización o \textit{dockers}} es una novedosa tecnología para la virtualización de software/servicios, frente a la virtualización de sistemas operativos (p.ej. máquinas virtuales). La orquestación de \textit{dockers} permite organizar complejos sistemas de información con muchas facilidades.
\item PostgreSQL/PostGIS es la \textit{geodatabase libre} más potente del mercado, destacando por sus opciones de \textbf{extensibilidad} (p.ej. PostGIS en sí misma es una extensión de PostgreSQL).
\end{itemize}
\end{graybox}



\subsection{Control de versiones}



\begin{itemize}
\item\textbf{Git}: 
\item\textbf{GitHub}: 
\end{itemize}



\begin{figure}
\begin{center}
\includegraphics[width=0.9\textwidth]{Metodologia/Figs/diary.png}
\caption{Flujo de proceso de actualización de ficheros. \label{fig:diary}}
\end{center}
\end{figure}



\begin{figure}
\begin{center}
\includegraphics[width=0.9\textwidth]{Metodologia/Figs/pullrequest.png}
\caption{Flujo de proceso de trabajo colaborativo entre repositorios. \label{fig:pullrequest}}
\end{center}
\end{figure}


\subsection{Contenerización y orquestación de servicios}





Se han utilizado los siguientes elementos:
\begin{itemize}
\item\textbf{Docker}: 
\item\textbf{Docker Hub}: 
\item\textbf{Docker-compose}: 
\end{itemize}




\begin{figure}
\begin{center}
\includegraphics[width=\textwidth]{Metodologia/Figs/ci.png}
\caption{Flujo de proceso de integración continua. \label{fig:ci}}
\end{center}
\end{figure}


\subsection{Extensibilidad}



\begin{itemize}
\item\textbf{PostgreSQL/PostGIS}: 
\end{itemize}

\subsection{Aplicaciones}



\begin{itemize}
\item\textbf{PgAdmin4}: 
\item\textbf{QGIS 2.18}: 
\end{itemize}

\section{Conjunto de datos}

\begin{graybox}
\begin{itemize}
\item En este trabajo se han utilizado dos conjuntos de datos, \textbf{un paisaje de ejemplo y el SIOSE de 2011 completo}, para poner a prueba la extensión \textit{pg\_landmetrics} propuesta en los objetivos.
\end{itemize}
\end{graybox}



\begin{figure}
\begin{center}
\includegraphics[width=\textwidth]{Metodologia/Figs/land_test.png}
\caption{Uso y cobertura del suelo del paisaje de ejemplo. \label{fig:lan_test}}
\end{center}
\end{figure}

% Please add the following required packages to your document preamble:
% \usepackage{booktabs}
\begin{table}[]
\centering
\caption{Atributos del primer conjunto de datos.}
\label{my-label}
\begin{tabular}{@{}lll@{}}
\toprule
\textbf{Nombre} & \textbf{Tipo de campo}   & \textbf{Descripción}                    \\ \midrule
gid             & integer                  & Identificador único de cada polígono    \\
geom            & geometry                 & Geometría del polígono                  \\
category        & character varying (text) & Clasificación de la cobertura del suelo \\
svg\_color      & character varying (text) & Color según tipo de categoría           \\ \bottomrule
\end{tabular}
\end{table}




\begin{table}[]
\centering
\caption{Color SVG según el tipo de categoría al que corresponde.}
\label{my-label}
\begin{tabular}{ll}
\hline
\textbf{Category}   & \textbf{svg\_color} \\ \hline
Agricultural area   & lightsalmon         \\
Continuous urban    & lightcoral          \\
Discontinuous urban & lightpink           \\
Forest              & forestgreen         \\
Main road           & gray                \\
Secondary road      & lightgrey           \\
River               & royalblue           \\
Water body          & mediumblue          \\ \hline
\end{tabular}
\end{table}









\section{Selección de métricas}

\begin{graybox}
\begin{itemize}
\item El número potencial de métricas del paisaje es indeterminado y depende de muchos factores (p.ej. objetivos del estudio, modelos de datos como \textit{raster/vector} o en red, niveles de agregación y/o escala, etc).
\item Resulta esencial determinar unas \textbf{métricas representativas} para esta primera propuesta.
\end{itemize}
\end{graybox}


% Please add the following required packages to your document preamble:
% \usepackage{multirow}
\begin{table}[]
\centering
\caption{Listado de métricas de paisaje disponibles en la extensión.}
\label{my-label}
\begin{tabular}{lll}
\hline
\textbf{Nivel}             & \textbf{Métrica}                     & \textbf{Abreviatura} \\ \hline
\multirow{8}{*}{Patch}     & Patch Area                           & AREA                 \\
                           & Patch Perimeter                      & PERIM                \\
                           & Perimeter-Area-Ratio                 & PARA                 \\
                           & Shape Index                          & SHAPE                \\
                           & Core Area                            & CORE                 \\
                           & Number of Core Areas                 & NCORE                \\
                           & Core Area Index                      & CAI                  \\
                           & Euclidean Nearest Neighbour Distance & ENN                  \\ \hline
\multirow{8}{*}{Class}     & Total (Class) Area                   & CA                   \\
                           & Percentage of Landscape              & PLAND                \\
                           & Total Edge                           & TE                   \\
                           & Edge Density                         & ED                   \\
                           & Total Core Area                      & TCA                  \\
                           & Core Area Percentage of Landscape    & CPLAND               \\
                           & Number of Patches                    & NP                   \\
                           & Patch Density                        & PD                   \\ \hline
\multirow{9}{*}{Landscape} & Total Area                           & TA                   \\
                           & Total Edge                           & TE                   \\
                           & Edge Density                         & ED                   \\
                           & Number of Patches                    & NP                   \\
                           & Patch Density                        & PD                   \\
                           & Patch Richness                       & PR                   \\
                           & Patch Richness Density               & PRD                  \\
                           & Shannon's Diversity Index            & SHDI                 \\
                           & Simpson's Diversity Index            & SHIDI                \\ \hline
\end{tabular}
\end{table}


\section{Implementación/desarrollo de funciones en PostgreSQL}

\begin{graybox}
\begin{itemize}
\item Los desarrollos en PostgreSQL se pueden realizar en lenguajes de programación como ANSI C, SQL y/o distintos lenguajes procedurales (p.ej. PLpgSQL, PL/R, PL/Python, entre muchos otros), \textbf{dependiendo de las necesidades}.
\end{itemize}
\end{graybox}


\lstset{caption=Crear una función para calcular el IDW (I),label= IDW1}
\begin{SQL}
SELECT St_Area(geom)/10000 FROM sample_patches_25830;
SELECT St_Area(geom)/10000 FROM sample_patches_4326;
\end{SQL}

\lstset{caption=Crear una función para calcular el IDW (I),label= IDW1}
\begin{SQL}
SELECT SUM(St_Area(geom))/10000, category FROM sample_patches_25830 GROUP BY category;
SELECT SUM(St_Area(geom))/10000, category FROM sample_patches_4326 GROUP BY category;
\end{SQL}

\lstset{caption=Crear una función para calcular el IDW (I),label= IDW1}
\begin{SQL}
SELECT SUM(St_Area(geom)) FROM sample_patches_25830;
SELECT SUM(St_Area(geom)) FROM sample_patches_4326;
\end{SQL}


\lstset{caption=Crear una función para calcular el IDW (I),label= IDW1}
\begin{SQL}
SELECT St_Distance(p1.geom, p2.geom) 
FROM sample_patches_25830 AS p1, sample_patches_25830 AS p2
WHERE p1.id = 1 AND p1.id <> p2.id AND p2.category= "category"
ORDER BY St_Distance (p1.geom, p2.geom)
LIMIT 1;

SELECT St_Distance(p1.geom, p2.geom) 
FROM sample_patches_4326 AS p1, sample_patches_4326 AS p2
WHERE p1.id = 1 AND p1.id <> p2.id AND p2.category= "category"
ORDER BY St_Distance (p1.geom, p2.geom)
LIMIT 1;
\end{SQL}

\lstset{caption=Crear una función para calcular el IDW (I),label= IDW1}
\begin{SQL}
SELECT SUM(St_Area(St_Buffer(geom, -100)))/10000 FROM sample_patches_25830 GROUP BY category;
SELECT SUM(St_Area(St_Buffer(geom, -100)))/10000 FROM sample_patches_4326 GROUP BY category;
\end{SQL}

\lstset{caption=Crear una función para calcular el IDW (I),label= IDW1}
\begin{SQL}
SELECT SUM(St_Perimeter(geom)/St_Area(geom))*10000 FROM sample_patches_25830;
SELECT SUM(St_Perimeter(geom)/St_Area(geom))*10000 FROM sample_patches_4326;
\end{SQL}


\lstset{caption=Crear una función para calcular el IDW (I),label= IDW1}
\begin{SQL}
SELECT (p_corearea(geom, 50)).value FROM sample_patches_25830;
SELECT (p_corearea(geom, 50)).value FROM sample_patches_4326;
\end{SQL}

\lstset{caption=Crear una función para calcular el IDW (I),label= IDW1}
\begin{SQL}
SELECT c_totalarea(geom,category) FROM sample_patches_25830;
SELECT c_totalarea(geom,category) FROM sample_patches_4326;
\end{SQL}




\section{Documentación de la extensión}

\begin{graybox}
\begin{itemize}
\item Una parte fundamental de esta metodología es la \textbf{documentación} del desarrollo y uso de la extensión. 
\item Una buena documentación con ejemplos facilitará el cálculo de métricas del paisaje en grandes repositorios, \textbf{sobretodo para aquellos usuarios con menos experiencia} en PostgreSQL/PostGIS.
\end{itemize}
\end{graybox}

A lo largo del trabajo, se aplican una serie de lenguajes de marcado para la documentación de la extensión. Pero antes se obtienen conocimientos previos sobre ellos y su funcionamiento para utilizarlos durante el proyecto. Los lenguajes utilizados son:
\begin{itemize}
\item\textbf{Markdown}\footnote{\url{https://github.com/adam-p/markdown-here/wiki/Markdown-Cheatsheet}} es un lenguaje ligero capaz de convertir texto plano a lenguaje HTML. Permite una escritura sencilla y conserva un diseño fácil de lectura. Es compatible con muchas plataformas. Este tipo de lenguaje es utilizado para documentar la extensión en la plataforma de GitHub y la descripción de cada una de las métricas de paisaje.
\item\textbf{TeX}\footnote{\url{https://www.latex-project.org/}} es el lenguaje que se utiliza en el sistema de textos LaTeX y que crea documentos con una alta calidad tipográfica. Se utiliza para escribir artículos o libros científicos, y desde hace tiempo este lenguaje se emplea por un gran número de usuarios. Este tipo de lenguaje es utilizado para redactar este trabajo. Para trabajar con este lenguaje, se utiliza la aplicación Texmaker que se ejecuta desde un contenedor Docker.
\item\textbf{Scalable Vector Graphics (SVG)} es un lenguaje capaz de crear gráficos basados en vectores escalables a partir de archivos vectoriales en 2D y en formato XML. En esta década muchos de los navegadores web utilizan este tipo de lenguaje para sus gráficos. Gracias a este lenguaje, los gráficos no pierden calidad, pueden ser escalables y ocupan menos espacio en la memoria. Este tipo de lenguaje es utilizado para crear las figuras del trabajo.

Por ello, se desarolla una función SQL, como primera propuesta, que dibuja y colorea los gráficos vectoriales que acompañan en el trabajo y en la documentación de la extensión en la plataforma GitHub. Para que las figuras tengan el color correspondiente al tipo de cobertura al que pertenecen, se aplica el código SVG que acompaña a cada polígono en la tabla de atributos del primer conjuntos de datos. La función SVG puede consultarse en el Anexo II.
\item\textbf{Mermaid}\footnote{\url{https://mermaidjs.github.io/}} es un lenguaje de secuencia, que utiliza etiquetas similares a las que son empleadas en el lenguaje de marcado, capaz de generar gráficos a partir de texto por medio de JavaScript. Este tipo de lenguaje se ha utilizado para crear los diagramas de secuencia y el diagrama de Gantt.
\end{itemize}

%!TEX root = ../thesis.tex
%*******************************************************************************
%****************************** Metodologia *********************************
%*******************************************************************************


\chapter{Metodología}\label{chap:metod}

Este trabajo aplica una metodología de \textbf{integración continua y desarrollo colaborativo} de acuerdo con los objetivos del proyecto SIOSE-INNOVA y con un marco de trabajo actual aplicado tanto en el Laboratorio de Geomática de la Universidad de Alicante como por otras empresas conocidas del sector (CARTO, Geographica, entre otros). En \citet{Zaragozi2017} se presentan distintos flujos de trabajo basados en esta metodología los cuales están muy relacionados con lo que se presenta en este capítulo.


\section{Integración continua y desarrollo colaborativo}

\begin{graybox}
\begin{itemize}
\item El trabajo colaborativo se ha coordinado utilizando \textit{Git} que es el sistema de \textbf{control de versiones} más popular de los últimos años (p.ej. utilizado en , PostGIS, QGIS, CARTO y decenas de proyectos ESRI, entre muchos otros).
\item La \textbf{contenerización o \textit{dockers}} es una novedosa tecnología para la virtualización de software/servicios, frente a la virtualización de sistemas operativos (p.ej. máquinas virtuales). La orquestación de \textit{dockers} permite organizar complejos sistemas de información con muchas facilidades.
\item PostgreSQL/PostGIS es la \textit{geodatabase libre} más potente del mercado, destacando por sus opciones de \textbf{extensibilidad} (p.ej. PostGIS en sí misma es una extensión de PostgreSQL).
\end{itemize}
\end{graybox}



\subsection{Control de versiones}


\begin{itemize}
\item\textbf{Git}: 
\item\textbf{GitHub}: 
\end{itemize}



\begin{figure}
\begin{center}
\includegraphics[width=0.9\textwidth]{Metodologia/Figs/diary.png}
\caption{Flujo de proceso de actualización de ficheros. \label{fig:diary}}
\end{center}
\end{figure}



\begin{figure}
\begin{center}
\includegraphics[width=0.9\textwidth]{Metodologia/Figs/pullrequest.png}
\caption{Flujo de proceso de trabajo colaborativo entre repositorios. \label{fig:pullrequest}}
\end{center}
\end{figure}


\subsection{Contenerización y orquestación de servicios}





Se han utilizado los siguientes elementos:
\begin{itemize}
\item\textbf{Docker}: 
\item\textbf{Docker Hub}: 
\item\textbf{Docker-compose}: 
\end{itemize}




\begin{figure}
\begin{center}
\includegraphics[width=\textwidth]{Metodologia/Figs/ci.png}
\caption{Flujo de proceso de integración continua. \label{fig:ci}}
\end{center}
\end{figure}


\subsection{Extensibilidad}



\begin{itemize}
\item\textbf{PostgreSQL/PostGIS}: 
\end{itemize}

\subsection{Aplicaciones}



\begin{itemize}
\item\textbf{PgAdmin4}: 

\begin{figure}
\begin{center}
\includegraphics[width=\textwidth]{Metodologia/Figs/carga-siose-2011.png}
\caption{Flujo de proceso de integración continua. \label{fig:carga}}
\end{center}
\end{figure}

\item\textbf{QGIS 2.18}: 
\end{itemize}

\section{Conjunto de datos}

\begin{graybox}
\begin{itemize}
\item En este trabajo se han utilizado dos conjuntos de datos, \textbf{un paisaje de ejemplo y el SIOSE de 2011 completo}, para poner a prueba la extensión \textit{pg\_landmetrics} propuesta en los objetivos.
\item Para demostrar que la \textbf{extensión} es \textbf{escalable}, se han utilizado distintas escalas de referencia a partir de \textit{grids}.
\end{itemize}
\end{graybox}

Para someter a prueba la extensión \textit{pg\_landmetrics} se han utilizados dos conjuntos de datos: un paisaje de ejemplo y el SIOSE de 2011.

El \textbf{paisaje de ejemplo} se ha utilizado para comprobar si las métricas de paisaje funcionan correctamente desde PgAdmin como también en QGIS. La escala de referencia es 1:50.000 en dos sistemas geodésicos de referencia: European Terrestrial Reference System 1989 (ETRS89) y World Geodetic System 84 (WGS84), ya que las métricas no se calculan igual si son proyectadas (\textit{geometry}) o son coordenadas esféricas (\textit{geography}). 

En este conjunto de datos hay un total de 51 polígonos que se han digitalizado a partir de las herramientas de geoproceso y edición avanzada mediante la aplicación de QGIS. En total hay 8 categorías de cobertura de suelo donde a cada polígono le corresponde un tipo de categoría representada por un color (ver figura \ref{fig:zona_andrea}). Se han definido los colores a partir de los 147 que presenta Scalable Vector Graphics (SVG) Specification\footnote{\url{http://www.december.com/html/spec/colorsvg.html}} además del apoyo de la clasificación de color que especifica el \textit{Corine Land Cover (CLC)}.

Una vez se han comprobado que las métricas de paisaje funcionan correctamente, el segundo conjuntos de datos que se ha utilizado para obtener los resultados es el \textbf{SIOSE 2011}, datos que se han obtenido a partir del equipo técnico del SIOSE que colabora en el proyecto SIOSE-INNOVA. 

Esta base de datos tan voluminosa se componen de 4 tablas. En la tabla de datos \texttt{t\_nomes} se encuentran todas las nomenclaturas de cada polígono, en \texttt{t\_poli\_atrib} todos los atributos de los polígonos como por ejemplo el código multietiqueta de clasificación del SIOSE o de clasificación de las distintas nomenclaturas (iberpix, perfil ambiental, toponimia, \textit{Corine Land Cover}, etc.), en cuanto a la tabla \texttt{t\_poli\_geo} ofrece la geometría de cada uno de los 2.562.800 de polígonos en total y finalmente, en la tabla \texttt{t\_valores} calculos estádisticos como por ejemplo la superficie en hectáreas.

Como se menciona en la sección \ref{sec:siose}, los polígonos del SIOSE presentan una \textit{cobertura simple} o una \textit{cobertura compuesta}. Dentro de esta última cobertura se encuentran las coberturas no predefindidas, es decir, que un mismo polígono puede contener varías coberturas de suelo. A raíz de esta característica, se ha realizado una reclasificación de los polígonos a partir de la cobertura prevalente que presenta la jerarquización de la clasificación de los usos y coberturas del suelo del SIOSE. Con ello se ha obtenido una nueva tabla con 85 coberturas prevalentes asignadas a cada uno de los polígonos. Para que las consultas sean más rápidas entre las distintas tablas se generaron índices para indexar las categorías. 

Una vez se han reclasificado todos los polígonos del SIOSE 2011, se han obtenido \textbf{\textit{grids}} (cuadrículas cartográficas) para el cálculo de métricas del paisaje en distintas escalas de referencia, a través del Centro de Descargas del Instituto Geográfico Nacional (CNIG)\footnote{\url{http://centrodedescargas.cnig.es/CentroDescargas/index.jsp}}. 

A partir del grid 1:500.000 se han escogido dos celdas de las zonas de estudio en las cuales se han calculado algunas métricas de paisaje. Estas dos zonas comprenden en su mayor parte a la provincia de Alicante y a la provincia de Zaragoza, y los alrededores de cada una. El motivo de por qué se han escogido estas dos zonas es por hacer referencia al lugar donde se han realizado las prácticas de empresa y dónde se ha impartido el \textit{máster}.

Dado que ya se han escogido las zonas de estudio, a continuación se han seleccionado y extraído todas aquellas celdas que comprenden las dos celdas de las áreas de estudio con escala de referencia 1:25.000, 1:50.000 y 1:100.000. A partir de éstas se han realizado todos los resultados que pueden consultarse en el capítulo \ref{chap:result}.

Todas las características técnicas de los conjuntos de datos que se han utilizado en este trabajo se encuentran en la tabla \ref{tab:datos}.

\begin{figure}
\begin{center}
\includegraphics[width=\textwidth]{Metodologia/Figs/zona_andrea.png}
\caption{Cobertura del suelo del paisaje de ejemplo. \label{fig:zona_andrea}}
\end{center}
\end{figure}

% Please add the following required packages to your document preamble:
% \usepackage{booktabs}
% \usepackage{multirow}
\begin{table}[]
\centering
\caption{Características de los conjuntos de datos utilizados\label{tab:datos}}
\begin{tabular}{@{}llll@{}}
\toprule
\textbf{Tipo}               & \textbf{Tablas} & \textbf{Filas} & \textbf{Tamaño total} \\ \midrule
\multirow{4}{*}{SIOSE-2011} & t\_nomes        & 36.790.972     & 6116 MB               \\
                            & t\_poli\_atrib  & 2.562.800      & 451 MB                \\
                            & t\_poli\_geo    & 2.562.800      & 3981 MB               \\
                            & t\_valores      & 10.932.639     & 1041 MB               \\ \midrule
\multirow{4}{*}{Grids}      & grid\_25k       & 756            & 232,3 kB              \\
                            & grid\_50k       & 192            & 57,8 kB               \\
                            & grid\_100k      & 48             & 13,8 kB               \\
                            & grid\_500k      & 2              & 677bytes              \\ \midrule
\multirow{2}{*}{Sample}     & sample\_25830   & 51             & 122,6 kB              \\
                            & sample\_4326    & 51             & 122,5 kB              \\ \bottomrule
\end{tabular}
\end{table}





\section{Selección de métricas}

\begin{graybox}
\begin{itemize}
\item El número potencial de métricas del paisaje es indeterminado y depende de muchos factores (p.ej. objetivos del estudio, modelos de datos como \textit{raster/vector} o en red, niveles de agregación y/o escala, etc).
\item Resulta esencial determinar unas \textbf{métricas representativas} para esta primera propuesta.
\end{itemize}
\end{graybox}

Como se indica en la sección \ref{sec:metrica}, existen cientos de métricas de paisaje que no siempre tienen significado para todas las aplicaciones de estudio ya que dependen de muchos factores, como por ejemplo los objetivos del estudio, el modelo de datos, la escala, nivel de agregación etc. Por este mismo motivo y según los objetivos específicados en este trabajo y en el proyecto SIOSE-INNOVA, se han seleccionado unas determinadas métricas de paisaje para esta primera propuesta de extensión.

En la tabla \ref{tab:listmetricas} se aprecian las métricas, acompañadas por su abreviatura, que se han escogido divididas en tres niveles de agregación: a nivel de polígono (\textit{patch}), nivel de categoría (\textit{class}) y nivel de paisaje (\textit{landscape}). Para ello se ha seleccionado un número equitativo entre los niveles de agregación. Además, se han querido escoger algunas métricas que calculen operaciones simples como por ejemplo el área o perímetro, y por otro lado, métricas cuyos cálculos sean más complejos como por ejemplo la distancia del vecino más próximo o la densidad, entre otros.

% Please add the following required packages to your document preamble:
% \usepackage{multirow}
\begin{table}[]
\centering
\caption{Métricas de paisaje disponibles en la extensión.}
\label{tab:listmetricas}
\begin{tabular}{lll}
\hline
\textbf{Nivel}             & \textbf{Métrica}                     & \textbf{Abreviatura} \\ \hline
\multirow{8}{*}{Patch}     & Patch Area                           & AREA                 \\
                           & Patch Perimeter                      & PERIM                \\
                           & Perimeter-Area-Ratio                 & PARA                 \\
                           & Shape Index                          & SHAPE                \\
                           & Core Area                            & CORE                 \\
                           & Number of Core Areas                 & NCORE                \\
                           & Core Area Index                      & CAI                  \\
                           & Euclidean Nearest Neighbour Distance & ENN                  \\ \hline
\multirow{8}{*}{Class}     & Total (Class) Area                   & CA                   \\
                           & Percentage of Landscape              & PLAND                \\
                           & Total Edge                           & TE                   \\
                           & Edge Density                         & ED                   \\
                           & Total Core Area                      & TCA                  \\
                           & Core Area Percentage of Landscape    & CPLAND               \\
                           & Number of Patches                    & NP                   \\
                           & Patch Density                        & PD                   \\ \hline
\multirow{9}{*}{Landscape} & Total Area                           & TA                   \\
                           & Total Edge                           & TE                   \\
                           & Edge Density                         & ED                   \\
                           & Number of Patches                    & NP                   \\
                           & Patch Density                        & PD                   \\
                           & Patch Richness                       & PR                   \\
                           & Patch Richness Density               & PRD                  \\
                           & Shannon's Diversity Index            & SHDI                 \\
                           & Simpson's Diversity Index            & SHIDI                \\ \hline
\end{tabular}
\end{table}


\section{Implementación/desarrollo de funciones en PostgreSQL}

\begin{graybox}
\begin{itemize}
\item Los desarrollos en PostgreSQL se pueden realizar en lenguajes de programación como ANSI C, SQL y/o distintos lenguajes procedurales (p.ej. PLpgSQL, PL/R, PL/Python, entre muchos otros), \textbf{dependiendo de las necesidades}.
\end{itemize}
\end{graybox}


\lstset{caption=Crear una función para calcular el IDW (I),label= IDW1}
\begin{SQL}
SELECT St_Area(geom)/10000 FROM sample_patches_25830;
SELECT St_Area(geom)/10000 FROM sample_patches_4326;
\end{SQL}

\lstset{caption=Crear una función para calcular el IDW (I),label= IDW1}
\begin{SQL}
SELECT SUM(St_Area(geom))/10000, category FROM sample_patches_25830 GROUP BY category;
SELECT SUM(St_Area(geom))/10000, category FROM sample_patches_4326 GROUP BY category;
\end{SQL}

\lstset{caption=Crear una función para calcular el IDW (I),label= IDW1}
\begin{SQL}
SELECT SUM(St_Area(geom)) FROM sample_patches_25830;
SELECT SUM(St_Area(geom)) FROM sample_patches_4326;
\end{SQL}


\lstset{caption=Crear una función para calcular el IDW (I),label= IDW1}
\begin{SQL}
SELECT St_Distance(p1.geom, p2.geom) 
FROM sample_patches_25830 AS p1, sample_patches_25830 AS p2
WHERE p1.id = 1 AND p1.id <> p2.id AND p2.category= "category"
ORDER BY St_Distance (p1.geom, p2.geom)
LIMIT 1;

SELECT St_Distance(p1.geom, p2.geom) 
FROM sample_patches_4326 AS p1, sample_patches_4326 AS p2
WHERE p1.id = 1 AND p1.id <> p2.id AND p2.category= "category"
ORDER BY St_Distance (p1.geom, p2.geom)
LIMIT 1;
\end{SQL}

\lstset{caption=Crear una función para calcular el IDW (I),label= IDW1}
\begin{SQL}
SELECT SUM(St_Area(St_Buffer(geom, -100)))/10000 FROM sample_patches_25830 GROUP BY category;
SELECT SUM(St_Area(St_Buffer(geom, -100)))/10000 FROM sample_patches_4326 GROUP BY category;
\end{SQL}

\lstset{caption=Crear una función para calcular el IDW (I),label= IDW1}
\begin{SQL}
SELECT SUM(St_Perimeter(geom)/St_Area(geom))*10000 FROM sample_patches_25830;
SELECT SUM(St_Perimeter(geom)/St_Area(geom))*10000 FROM sample_patches_4326;
\end{SQL}


\lstset{caption=Crear una función para calcular el IDW (I),label= IDW1}
\begin{SQL}
SELECT (p_corearea(geom, 50)).value FROM sample_patches_25830;
SELECT (p_corearea(geom, 50)).value FROM sample_patches_4326;
\end{SQL}

\lstset{caption=Crear una función para calcular el IDW (I),label= IDW1}
\begin{SQL}
SELECT c_totalarea(geom,category) FROM sample_patches_25830;
SELECT c_totalarea(geom,category) FROM sample_patches_4326;
\end{SQL}




\section{Documentación de la extensión}

\begin{graybox}
\begin{itemize}
\item Una parte fundamental de esta metodología es la \textbf{documentación} del desarrollo y uso de la extensión. 
\item Una buena documentación con ejemplos facilitará el cálculo de métricas del paisaje en grandes repositorios, \textbf{sobretodo para aquellos usuarios con menos experiencia} en PostgreSQL/PostGIS.
\end{itemize}
\end{graybox}

La documentación del desarrollo y uso de la extensión es una de las partes más importante de la metodología de este trabajo. Una buena documentación facilita la aplicación de todas las medidas necesarias para llevar a cabo el funcionamiento de la extensión a cualquier usuario, sobretodo a aquellos menos expertos en la materia. Así pues, se han aplicado los siguientes lenguajes de marcado:

\begin{itemize}
\item\textbf{Markdown}\footnote{\url{https://github.com/adam-p/markdown-here/wiki/Markdown-Cheatsheet}} es un lenguaje ligero que permite una escritura sencilla y de fácil lectura usando texto plano. Se ha utilizado para documentar el usp de la extensión.
\item\textbf{TeX}\footnote{\url{https://www.latex-project.org/}} es el lenguaje que se utiliza en el sistema de textos LaTeX y que crea documentos con una alta calidad tipográfica. Desde hace tiempo este lenguaje se emplea por un gran número de usuarios para escribir artículos o libros científicos. Para trabajar con este lenguaje, se ha utilizado la aplicación Texmaker y se ha utilizado para escribir este trabajo.
\item\textbf{Scalable Vector Graphics (SVG)}\footnote{\url{https://www.w3schools.com/graphics/svg_intro.asp}} es un lenguaje capaz de crear gráficos basados en vectores escalables de alta calidad de resolución. A partir de este lenguaje se ha desarrollado una función capaz de representar gráficos vectoriales a partir de las geometrías de cualquier geodatabase (ver en el anexo \ref{anex:svg}).
\item\textbf{Mermaid}\footnote{\url{https://mermaidjs.github.io/}} es un lenguaje que genera gráficos a partir de texto mediante JavaScript. Se han generado desde diagramas de flujo hasta diagramas de secuencia y de Gantt.
\end{itemize}

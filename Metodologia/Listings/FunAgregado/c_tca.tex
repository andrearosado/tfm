\lstset{caption=Función de agregado para calcular TCA por categorías.,label=tca2}
\begin{SQL}
CREATE OR REPLACE FUNCTION c_totalcorearea_state(
	current_state metric_labeled,
	geom geometry,
	category text)
    RETURNS metric_labeled
    LANGUAGE 'sql'

AS 
$BODY$

WITH inputs AS (
    	SELECT current_state AS cstate
), melt AS (
    	SELECT unnest((cstate).pairs) AS m2 FROM inputs
    	UNION 
	SELECT (category, (p_corearea(geom)).value)::metric_labeled_pair AS m2
), summarize AS (
	SELECT (m2).label, SUM((m2).value) AS value FROM melt GROUP BY (m2).label
)
SELECT (13, ARRAY(SELECT (label, value)::metric_labeled_pair FROM summarize))::metric_labeled;

$BODY$;

CREATE AGGREGATE c_totalcorearea(geometry, text)(
    SFUNC=c_totalcorearea_state,
    STYPE=metric_labeled,
    INITCOND='(13,{})'
);

COMMENT ON AGGREGATE c_totalcorearea(geom geometry, category text) IS 'Suma las áreas de los núcleos de cada polígono de la misma categoría dividido por 10.000 para devolver un valor en Hectáreas.';
\end{SQL}




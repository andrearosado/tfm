%!TEX root = ../thesis.tex
%*******************************************************************************
%*********************************** Introduccion *****************************
%*******************************************************************************

\chapter{Introducción}\label{chap:intro}

\begin{graybox}
\begin{itemize}
\item El SIOSE es una valiosa \textbf{base de datos de ocupación del suelo} que contiene un gran volumen de información territorial de toda España.
\item Desde su aparición en 2005, SIOSE se ha convertido en un repositorio de referencia para sus homólogos europeos, llegando a ser un \textbf{modelo para la iniciativa EAGLE} (\textit{SIOSE europeo}). 
\item A pesar de su gran potencial, el SIOSE presenta ciertos problemas de \textit{usabilidad} debidos a su gran volumen y complejidad (p.ej. desde aplicaciones SIG de escritorio).
\item Las \textbf{métricas de paisaje} son métodos cuantitativos que sirven para analizar la estructura del paisaje y otros fenómenos (p.ej. evolución del paisaje, conectividad de ecosistemas, entre otros).
\item FRAGSTATS, Conefor Sensinode, Patch Analyst, entre otros, son aplicaciones de escritorio muy utilizadas para el cálculo de métricas del paisaje. No obstante, \textbf{no hay ninguna aplicación} que sea fácilmente \textbf{escalable y extensible} como para realizar análisis sobre una geodatabase similar a la del SIOSE.
\item El principal objetivo de este trabajo es \textbf{desarrollar una extensión} (PostgreSQL/PostGIS) capaz de calcular métricas de paisaje a partir de geodatabases tan voluminosas o más como la del SIOSE \textit{actual} (2014).
\end{itemize}
\end{graybox}

\ifpdf
    \graphicspath{{Chapter1/Figs/Raster/}{Chapter1/Figs/PDF/}{Chapter1/Figs/}}
\else
    \graphicspath{{Chapter1/Figs/Vector/}{Chapter1/Figs/}}
\fi


\section{Análisis de la estructura del paisaje a través del SIOSE}


El SIOSE caracteristicas

El SIOSE importancia/alcance

Aplicaciones de los usos/coberturas del suelo...

Usabilidad del SIOSE. En jornada de girona (2011)

SIOSE-INNOVA



\section{Métricas de paisaje y software que las calcula} 

Métricas del paisaje como técnica/metodología para el estudio del paisaje... aplicaciones agua, biodiversidad, riesgos naturales (), estructura urbana,  etc

Software.  En \cite{Zaragozi2012} se establece una comparativa entre programas específicos para el cálculo de métricas del paisaje, destacando entre ellos FRAGSTATS. Otros programas como... Evidentemente, esta lista no puede estar completa. Existen muchos otros programas especializados que permiten el cálculo de este tipo de métricas/índices

También es posible calcular determinadas métricas con herramientas típicas de un SIG...




\section{Objetivos}


